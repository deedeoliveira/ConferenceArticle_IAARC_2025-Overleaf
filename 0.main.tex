\documentclass[preprint,12pt]{elsarticle}
%\documentclass[preprint,review,12pt]{elsarticle}
%\documentclass[final,1p,times]{elsarticle}

\usepackage{preamble}
\usepackage{array}
\usepackage{booktabs}
\usepackage{multirow}
\usepackage{graphicx}
\usepackage{tabularx}
\usepackage{subcaption}
\usepackage{float}
%\usepackage{lineno}
%\linenumbers
%\usepackage{background}
%\backgroundsetup{contents=Confidential}

\setlength{\arrayrulewidth}{0.1mm}

\journal{Journal XXXXXXXX}

\begin{document}

\tableofcontents

\newpage

\begin{frontmatter}

\title{BIM-FM interoperability: integrating IFC models' visualization and existing FM platforms}

\author[inst1]{Andressa Oliveira}
\author[inst1]{José Granja}
\author[inst2]{Pedro Machado}
\author[inst3]{Ali Motamedi}
\author[inst1]{Miguel Azenha}

\affiliation[inst1]
            {organization={University of Minho, ISISE, ARISE, Department of Civil Engineering},
            city={Guimarães},
            country={Portugal}}

\affiliation[inst2]
            {organization={Matosinhos City Council},
            city={Matosinhos},
            country={Portugal}}

\affiliation[inst3]
            {organization={Université du Québec à Montreal, École de Technologie Supérieure},
            city={Newcastle},
            country={United Kingdom}}
        
\begin{abstract}
The adoption of Building Information Modelling (BIM) in the operational phase of buildings is still restricted. One of the factors hindering adoption of the methodology is the lack of BIM capabilities in Facility Management (FM) platforms, such as three-dimensional visualisation of the buildings to be managed. This article presents a solution designed in collaboration with the Matosinhos City Council in Portugal to enable the management of its assets using the BIM methodology. The purpose of the solution is to integrate the current management platform used by the council (Infraspeak) with three-dimensional visualisation resources, and to allow the consultation and manipulation of operational data directly and integrated with this visualisation. The use of information models (BIM models) that follow the IFC (Industry Foundation Classes) scheme has been considered for asset visualisation. In this context, an integrated and customised web platform was developed using the IFCjs library to manipulate, investigate and visualise IFC files. In addition, the platform allows direct connection to the Infraspeak database via its Application Programming Interface (API). This article details the development process of the integration platform, its key components and its interoperability with Infraspeak. The solution developed is an innovative approach to the current limitations in operations management and supports the wider adoption of BIM for the FM area.
\end{abstract}

\begin{keyword}
%% keywords here, in the form: keyword \sep keyword
Facility Management (FM) \sep Building Information Modelling (BIM) \sep Interoperability \sep Integration
\end{keyword}

\end{frontmatter}

%This command determines from where the lies will start to be numbered
%\linenumbers

\section{Introduction}
\label{sec:introduction}

The operational phase of a built asset represents the largest part of its life cycle, totalling approximately 60\% of the costs associated with it [1]. Facility Management (FM) is the discipline responsible for the functionality, comfort and safety of facilities in the built environment [2], and is one of the fastest growing sectors in the construction industry [3]. With the industrialisation of this sector [4], asset managers have become increasingly interested in implementing the BIM methodology to support operations management [3]. This integrated information management methodology can bring numerous advantages to a sector that has to deal with a large amount and variety of data from different sources [5]. Due to the variety and complexity of assets that public buildings and/or large buildings can present [3], the implementation of the BIM methodology can bring great benefits in supporting operations managers looking to improve the efficiency of FM processes in these types of buildings. Despite the various advantages found in the literature, there is still resistance to adopting BIM for asset operation [6], and managers are faced with a built heritage that is already in the operational phase but is not yet managed, for the most part, with the support of the BIM methodology. The current scenario shows that the use of management platforms that assist in the operation of built assets, such as Computerized Maintenance Management System (CMMS), Computer Aided Facility Management (CAFM) and Intelligent Maintenance Management Platform (IMMP), is already widespread [4], [5]. However, current management models are not prepared for the rapid adoption of BIM because they are heavily based on traditional methods, such as the use of physical documents, even when computerised platforms are used [5]. For a wider and more efficient adoption of BIM in the operational sector, it is necessary to spread knowledge about the methodology's information management processes, following the guidelines of ISO 19650-3 [7], and thus adapting management models so that they can cope with emerging technologies [6]. This updating of the management model includes, in addition to processes, the use of platforms capable of integrating different databases, including information models [1]. In the BIM methodology, information models are the main source of information on an asset and can include, among other things, information blocks containing the three-dimensional geometric representation of the asset [8]. In this context, information models must be prepared in such a way as to be able to integrate with the management platforms used by operations managers. From this context emerges the need to support the implementation of the BIM methodology in operations management processes, with a focus on integration between different data sources. Studies in this area could have an impact on the adoption of BIM in the operational phase of assets, especially in the public sector.
From this scenario emerged a partnership between the Matosinhos City Council (CMMatosinhos), in Portugal, and the University of Minho (UMinho). The aim of this partnership is to support the management and operation of municipal assets by integrating the BIM methodology into the management platform already used by CMMatosinhos, Infraspeak. This platform is an IMMP and allows users to interact with a web interface to access its functionalities, but does not allow visualisation of the buildings and assets managed. The IMMP also allows access to its functionalities via an Application Programming Interface (API). Therefore, the ultimate goal of this work is to develop a web platform capable of integrating the IMMP database with the three-dimensional visualisation of the assets to be managed, allowing the consultation, editing and insertion of information associated with these assets. From the developments of this partnership, the management and operations area of CMMatosinhos will be able to add geometric information to the decision-making process (e.g. with the visual distribution of spaces/equipment in need of maintenance actions within a building, it will be possible to define more efficient action routes).
efficient). The integration platform to be developed needs to allow communication between two different databases: the IMMP database, which contains the information needed to operate the assets; and the block of information containing the geometry of the buildings and assets to be managed. For the sake of simplicity, the block of information with three-dimensional visualisation will be referred to as the model throughout this article.
The first stage in the development of this study (section 2) included defining the objectives of CMMatosinhos and the functionalities of the integration platform to be developed. After defining the functional requirements, it was possible to establish the interaction processes between the different databases (section 3.1) and define the information exchange requirements (EIR) to guide the development of the models that will be accessed from the developed platform (section 3.2). The EIR was developed in a simplified way and the methodology of the level of information required [9] was used to define the requirements. Subsequently, the integration platform was implemented, with tests carried out and the use of an example model (prototype) developed from the requirements defined previously (section 3.3). Finally, the conclusions and discussions relevant to the research carried out are summarised in section 4.
%\pagebreak
\section{Methodology and functional requirements for the platform}
\label{sec:methodology}
The partnership established requires the development of a platform capable of integrating the IMMP used by CMMatosinhos with the ability to visualise and manipulate three-dimensional models of buildings to be managed. The aim is to define information requirements (EIR) to guide the development of models to be used within this platform, taking into account buildings already in the operational phase (R.1 Figure 1), as well as those yet to be built (R.2 Figure 1). The platform should include two different user profiles (manager and ordinary). The manager-type user should be able to register buildings on the platform (G.1 Figure 1) and access information on buildings that have already been registered (G.2 Figure 1). The ordinary user, in turn, can access buildings that have already been registered (C.1 Figure 1). As the article was developed before the partnership was finalised, only the developments relating to interaction with buildings already registered from the manager profile (G.2 Figure 1) and the information requirements relating to existing buildings (R.1 Figure 1) are covered here.

\hl{FIGURA 1}

Considering access to existing and already registered buildings from the manager user profile, the end user's objectives were listed from a series of meetings with the head of the municipal buildings division of Matosinhos City Council (end user). From these meetings, six main objectives were defined, which were later rewritten as functionalities to be implemented in the platform to be developed (Table 1). With a clear definition of the platform's functionalities, it was decided that it would be made up of two different web pages: building selection and management area. ISO19650-3 [7], for asset operation, states that the development of the EIR should be derived from organisational (OIR) and operational (AIR) requirements. OIR and AIR were not developed for this work, and the objectives defined in Table 1 served as premises for the development of the simplified EIR.

\hl{TABELA 1}

The development of the web platform can be divided into two components, frontend and backend. For the frontend, HTML, CSS and JavaScript were used. Among the platform's functionalities, those related to visualising and reading the model were developed with the support of the IFCjs open source library. The use of this library allows the creation of customised and interactive viewers, which include viewing the entire building or subsets of it, as well as selecting specific elements within the model. In addition, the library allows you to access element properties and extract information from them. The backend was developed in Python, with the Flask web framework. The prototype building model was developed on the Autodesk Revit 2023 platform and followed the information requirements defined previously. This model was then exported following the IFC scheme (IFC4 ADD2 TC1) [10].
%\pagebreak
\section{Implementation}
\label{sec:implementation}

For the development of this work, two main data sources are considered that need to be integrated through the platform: the IMMP platform database and the model. The IFC models already registered, in turn, would be hosted in a second database referred to here as the "CMMatosinhos database". Within the databases, and considering the functionalities to be implemented (Table 1), three types of element are essential to the platform's objectives: locations, equipment and requests. The first type of element represents spatial assets, being the locations (or spaces) that make up the buildings managed by CMMatosinhos, essential to the spatial location of the equipment. Equipment, in turn, represents any physical asset that is part of the inventory of CMMatosinhos (vehicles, chairs, ladders, etc). Finally, requests represent the request for an operation action to be carried out, and are associated with the other types of element (equipment and location). Figure 2 shows the flow of information between the different databases, their elements and the platform components (frontend and backend). The main components of the information flow in Figure 2 are the IMMP identifiers and CMMatosinhos identifiers. The identifiers are unique keys for identifying the elements within the IMMP database and the CMMatosinhos inventory, respectively. These identifiers function as the interaction key capable of associating the same asset from the different sources (IMMP and CMMatosinhos inventory). The process of implementing the web platform is described below, including defining the identifiers for the IMMP database, asset identifiers within the CMMatosinhos inventory, developing the information requirements for existing building models and other aspects of the platform's development.

\hl{FIGURA 2}

\subsection{Definition of unique identifiers}
\label{subsec:identifiers}
\hl{acho que definition não é o melhor nome}

The unique identifiers were defined by analysing the IMMP database. Firstly, IMMP identifiers were defined, containing keys capable of uniquely identifying an element within this database. For the CMMatosinhos identifiers, the analysis was carried out using the IMMP database, but now to find attributes containing information relating to the identification of assets from the manager's organisation. Figure 3 shows the structure of the assets within the IMMP database, including the endpoints for accessing the information shown, attributes relevant to the development of the platform, the relationship between attributes of different elements and unique identifiers.

\hl{FIGURA 3}

\emph{IMMP identifier}
Within the IMMP database, sites are organised hierarchically and have the attribute "local\_id" as their unique identifier. On the other hand, registered assets (ELEMENT) are classified into two categories, EQUIPMENT and LOCAL. In this case, the locations included as elements are those at the lowest level of the registered hierarchy and are equivalent to individual spaces (room, changing room, ...). All elements are uniquely identified by the "element\_id" attribute. The numbering contained in these attributes ("local\_id" and "element\_id") is generated automatically by the IMMP, and does not correspond to the CMMatosinhos inventory. Therefore, the "local\_id" and "element\_id" attributes are considered to be the IMMP identifiers for locations and equipment, respectively. Finally, when analysing the elements of type request, named as failure within the IMMP database, we focused on those considered "open," e.g. that still require any kind of action. As this is not a physical asset, in addition to the attributes of the element itself, its relationship with equipment and locations was also analysed. Requests are uniquely identified using the "failure\_id" attribute (IMMP identifier), and contain the "local\_id" attribute capable of identifying the location to which the request is associated. On the other hand, information about its association with a device can only be accessed from the relationships of the device type elements, where the "failure\_id" numbers can be visualised.

\emph{CMMatosinhos identifier}
Requests do not have a CMMatosinhos identifier as they do not represent physical assets. As for locations, the value of the "full\_code" attribute represents the nomenclature used by CMMatosinhos to identify spaces hierarchically, and represents a unique code for each location. As for equipment, the value of the "nfc\_code" attribute was selected as a potential unique identifier. However, not all equipment contains associated NFC codes, which implies that it could not be used as the identification key for this type of asset. In a second analysis, it was decided to use the "code" attribute for this purpose. Within the CMMatosinhos inventory organisation, the value of this attribute is filled in using coding specific to the type of equipment (chair, vehicle, etc.). However, the end user clarified that this coding might not be unique for some specific types of equipment. Finally, it was concluded that if equipment had the same code, it would not be in the same space. Therefore, the combination of the attributes "code" of the active piece of equipment and "full\_code" of the location where it is located associated within the model, was defined as the CMMatosinhos identifier for the equipment. Finally, using the necessary level of information methodology (section 3.2), these identifiers were organised as properties that should be associated with the model elements. In Table 2, the term "Code" was used to represent these identifiers.

\subsection{Level of information need}
\label{subsec:loin}

In order to define the level of information required, it was considered that the model should allow visualisation of the building and assets to be managed, and contain only alphanumeric information relating to general asset information and that required for integration with the IMMP database (CMMatosinhos identifiers). As for the elements to be modelled (Table 2), three main groups of objects were defined: architecture, equipment and locations.

\hl{TABELA 2}

In this case, the architecture would serve the purpose of visualisation and only the geometric information aspects of the necessary level of information were required for this group. Of the three types of elements analysed in the previous subsection, applications were excluded as their visualisation was not required. For premises and equipment, geometric and alphanumeric information requirements were defined. As for the building as a whole (Table 3), since its geometry was already covered by the previous objects, only alphanumeric information requirements were defined. Non-applicable aspects are not shown in the tables.

\hl{TABELA 3}

\subsection{Platform development}
\label{subsec:platform}

The development of the platform, the tests carried out and the meetings with the end user took place iteratively and the platform's capabilities were gradually expanded. During development, it was possible to identify the limitations of the IMMP API and improve the information exchange processes. One of the optimisations implemented was the mapping between all the objects in the model (equipment or sites) and their respective open requests (open failures) (visible in Figure 2 in the backend). This mapping has been programmed to occur automatically as soon as the building is selected, and generates a direct association between the unique identifiers of the elements ("element\_id" and "local\_id") and the associated "failures\_id". This mapping also includes information such as the degree of prioritisation and the status of the request (failure). The decision was made because this information needs to be quickly visualised by the user and updated when they move between selected elements, which ultimately improves the end-user experience.

Figure 3 shows the visualisation of the integration platform accessible by the end user. The screenshot is of the management area, with visualisation enabled by floor and after selecting an element. The management area includes an upper section (Figure 4 a) with data relating to the building's identification and the possibility of editing the visualisation mode: whether by floor or entire building, whether all elements or only those with open requests. The next section (Figure 4 b) displays a warning about the presence and number of open requests associated with the building. The side menu (Figure 4 c) relates to asset information and is inserted into the user interface when a piece of equipment or location is selected. In this menu, it is possible to view information identifying the selected asset, the number of open requests associated with the asset, access the redirection to the asset's page within IMMP and the creation of orders associated directly with the asset, which are automatically registered within the IMMP platform.

\hl{FIGURA 4}





   
%\pagebreak
\section{Discussion and conclusion}
\label{sec:conclusion}

This work supports the implementation of the BIM methodology in the context of operations management, specifically in the area of integrating the methodology into existing management platforms. Challenges not yet foreseen were encountered during development. In particular, in terms of defining unique identifiers within CMMatosinhos' inventory, the coding used to organise its inventory was not functional for integration with other databases. This was because the existing coding did not uniquely identify a single asset, which meant that elements had to be identified using a combination of two codes (element id and local id). The iterative process used, with periodic meetings with CMMatosinhos, allowed these and other limitations to be overcome. Throughout the process, some optimisations were necessary to improve the user experience when interacting with the platform. For example, the automated mapping between elements and requests had an impact on the final product by reducing the user's waiting time for information to be updated. It should be noted that the IFCjs library and its great capacity for customisation are of great support to the development of web platforms in the context of using the BIM methodology. Through the library, the platform developed can be customised to the end user's objectives. In the context of this work, changes to visualisations (building, floor and elements with open requests) could be implemented, and it was also possible to investigate the model to extract the information needed for the integration platform to work. Finally, the work carried out advances knowledge of the implementation of BIM for building operations by showing a real case in which the methodology can be applied to an existing management system by integrating the information model with a management platform already used by the building manager.

\hl{erro}

operations. The decision-making process presented can be used to implement other integration processes to be developed in the building operations sector.


%\pagebreak
\section{Acknowledgements}
\label{sec:acknowledgements}

This work was partially funded by FCT/MCTES through national funds (PIDDAC) within the scope of the R\&D unit of the Institute for Sustainability and Innovation in Structural Engineering (ISISE), under reference UIDB/04029/2020 (doi.org/10.54499/UIDB/04029/2020), and within the scope of the ARISE Associated Laboratory for Advanced Production and Intelligent Systems under reference LA/P/0112/2020. This work was also funded by the doctoral scholarship (PRT/BD/154416/2023) awarded to the first author by the Foundation for Science and Technology (FCT), under the MIT Portugal Programme.
%\pagebreak

%\appendix
%\input{appendix}

\bibliographystyle{elsarticle-num} 
\bibliography{library}
 
\end{document}
\endinput
