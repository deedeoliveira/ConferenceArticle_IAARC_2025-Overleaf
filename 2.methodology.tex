\section{Methodology and functional requirements for the platform}
\label{sec:methodology}
The partnership established requires the development of a platform capable of integrating the IMMP used by CMMatosinhos with the ability to visualise and manipulate three-dimensional models of buildings to be managed. The aim is to define information requirements (EIR) to guide the development of models to be used within this platform, taking into account buildings already in the operational phase (R.1 Figure 1), as well as those yet to be built (R.2 Figure 1). The platform should include two different user profiles (manager and ordinary). The manager-type user should be able to register buildings on the platform (G.1 Figure 1) and access information on buildings that have already been registered (G.2 Figure 1). The ordinary user, in turn, can access buildings that have already been registered (C.1 Figure 1). As the article was developed before the partnership was finalised, only the developments relating to interaction with buildings already registered from the manager profile (G.2 Figure 1) and the information requirements relating to existing buildings (R.1 Figure 1) are covered here.

\hl{FIGURA 1}

Considering access to existing and already registered buildings from the manager user profile, the end user's objectives were listed from a series of meetings with the head of the municipal buildings division of Matosinhos City Council (end user). From these meetings, six main objectives were defined, which were later rewritten as functionalities to be implemented in the platform to be developed (Table 1). With a clear definition of the platform's functionalities, it was decided that it would be made up of two different web pages: building selection and management area. ISO19650-3 [7], for asset operation, states that the development of the EIR should be derived from organisational (OIR) and operational (AIR) requirements. OIR and AIR were not developed for this work, and the objectives defined in Table 1 served as premises for the development of the simplified EIR.

\hl{TABELA 1}

The development of the web platform can be divided into two components, frontend and backend. For the frontend, HTML, CSS and JavaScript were used. Among the platform's functionalities, those related to visualising and reading the model were developed with the support of the IFCjs open source library. The use of this library allows the creation of customised and interactive viewers, which include viewing the entire building or subsets of it, as well as selecting specific elements within the model. In addition, the library allows you to access element properties and extract information from them. The backend was developed in Python, with the Flask web framework. The prototype building model was developed on the Autodesk Revit 2023 platform and followed the information requirements defined previously. This model was then exported following the IFC scheme (IFC4 ADD2 TC1) [10].