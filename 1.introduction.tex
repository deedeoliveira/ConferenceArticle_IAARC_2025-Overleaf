\section{Introduction}
\label{sec:introduction}

The operational phase of a built asset represents the largest part of its life cycle, totalling approximately 60\% of the costs associated with it [1]. Facility Management (FM) is the discipline responsible for the functionality, comfort and safety of facilities in the built environment [2], and is one of the fastest growing sectors in the construction industry [3]. With the industrialisation of this sector [4], asset managers have become increasingly interested in implementing the BIM methodology to support operations management [3]. This integrated information management methodology can bring numerous advantages to a sector that has to deal with a large amount and variety of data from different sources [5]. Due to the variety and complexity of assets that public buildings and/or large buildings can present [3], the implementation of the BIM methodology can bring great benefits in supporting operations managers looking to improve the efficiency of FM processes in these types of buildings. Despite the various advantages found in the literature, there is still resistance to adopting BIM for asset operation [6], and managers are faced with a built heritage that is already in the operational phase but is not yet managed, for the most part, with the support of the BIM methodology. The current scenario shows that the use of management platforms that assist in the operation of built assets, such as Computerized Maintenance Management System (CMMS), Computer Aided Facility Management (CAFM) and Intelligent Maintenance Management Platform (IMMP), is already widespread [4], [5]. However, current management models are not prepared for the rapid adoption of BIM because they are heavily based on traditional methods, such as the use of physical documents, even when computerised platforms are used [5]. For a wider and more efficient adoption of BIM in the operational sector, it is necessary to spread knowledge about the methodology's information management processes, following the guidelines of ISO 19650-3 [7], and thus adapting management models so that they can cope with emerging technologies [6]. This updating of the management model includes, in addition to processes, the use of platforms capable of integrating different databases, including information models [1]. In the BIM methodology, information models are the main source of information on an asset and can include, among other things, information blocks containing the three-dimensional geometric representation of the asset [8]. In this context, information models must be prepared in such a way as to be able to integrate with the management platforms used by operations managers. From this context emerges the need to support the implementation of the BIM methodology in operations management processes, with a focus on integration between different data sources. Studies in this area could have an impact on the adoption of BIM in the operational phase of assets, especially in the public sector.
From this scenario emerged a partnership between the Matosinhos City Council (CMMatosinhos), in Portugal, and the University of Minho (UMinho). The aim of this partnership is to support the management and operation of municipal assets by integrating the BIM methodology into the management platform already used by CMMatosinhos, Infraspeak. This platform is an IMMP and allows users to interact with a web interface to access its functionalities, but does not allow visualisation of the buildings and assets managed. The IMMP also allows access to its functionalities via an Application Programming Interface (API). Therefore, the ultimate goal of this work is to develop a web platform capable of integrating the IMMP database with the three-dimensional visualisation of the assets to be managed, allowing the consultation, editing and insertion of information associated with these assets. From the developments of this partnership, the management and operations area of CMMatosinhos will be able to add geometric information to the decision-making process (e.g. with the visual distribution of spaces/equipment in need of maintenance actions within a building, it will be possible to define more efficient action routes).
efficient). The integration platform to be developed needs to allow communication between two different databases: the IMMP database, which contains the information needed to operate the assets; and the block of information containing the geometry of the buildings and assets to be managed. For the sake of simplicity, the block of information with three-dimensional visualisation will be referred to as the model throughout this article.
The first stage in the development of this study (section 2) included defining the objectives of CMMatosinhos and the functionalities of the integration platform to be developed. After defining the functional requirements, it was possible to establish the interaction processes between the different databases (section 3.1) and define the information exchange requirements (EIR) to guide the development of the models that will be accessed from the developed platform (section 3.2). The EIR was developed in a simplified way and the methodology of the level of information required [9] was used to define the requirements. Subsequently, the integration platform was implemented, with tests carried out and the use of an example model (prototype) developed from the requirements defined previously (section 3.3). Finally, the conclusions and discussions relevant to the research carried out are summarised in section 4.