\section{Discussion and conclusion}
\label{sec:conclusion}

This work supports the implementation of the BIM methodology in the context of operations management, specifically in the area of integrating the methodology into existing management platforms. Challenges not yet foreseen were encountered during development. In particular, in terms of defining unique identifiers within CMMatosinhos' inventory, the coding used to organise its inventory was not functional for integration with other databases. This was because the existing coding did not uniquely identify a single asset, which meant that elements had to be identified using a combination of two codes (element id and local id). The iterative process used, with periodic meetings with CMMatosinhos, allowed these and other limitations to be overcome. Throughout the process, some optimisations were necessary to improve the user experience when interacting with the platform. For example, the automated mapping between elements and requests had an impact on the final product by reducing the user's waiting time for information to be updated. It should be noted that the IFCjs library and its great capacity for customisation are of great support to the development of web platforms in the context of using the BIM methodology. Through the library, the platform developed can be customised to the end user's objectives. In the context of this work, changes to visualisations (building, floor and elements with open requests) could be implemented, and it was also possible to investigate the model to extract the information needed for the integration platform to work. Finally, the work carried out advances knowledge of the implementation of BIM for building operations by showing a real case in which the methodology can be applied to an existing management system by integrating the information model with a management platform already used by the building manager.

\hl{erro}

operations. The decision-making process presented can be used to implement other integration processes to be developed in the building operations sector.

